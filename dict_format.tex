% dict_format.tex
% Author: Hussein Esmail
% Created: 2022 06 08

% =========================== Variables ===========================
\def\myAuthor			{Hussein Esmail} % Author of this file
\def\mySubject			{PDF Subject Here}  % Used for Metadata
\def\myKeywords			{PDF Keywords Here} % Separated by comma
\def\myDateCreated		{2022 06 08}
\def\myDateUpdated		{\today}
\def\myTitle			{YorkU Class Scraper Dictionary Format}
% =========================== Variables ===========================

% =========================== Shortcuts ===========================
\def\bg{backgroundcolor}
\def\len{\setlength}
\newcommand{\image}[1]{[\textbf{IMAGE MISSING #1}]} % #1 is the descriptor about the image
\newenvironment{itemize*}{\begin{itemize}\setlength{\itemsep}{0pt}\setlength{\parskip}{0pt}}{\end{itemize}}
\newenvironment{enumerate*}{\begin{enumerate}\setlength{\itemsep}{0pt}\setlength{\parskip}{0pt}}{\end{enumerate}}
\newenvironment{enumalph*}{\begin{enumerate}[label=\alph*.]\setlength{\itemsep}{0pt}\setlength{\parskip}{0pt}}{\end{enumerate}}
\newenvironment{enumq*}{\begin{enumerate}[label=Q{{\arabic*}}.]\setlength{\itemsep}{0pt}\setlength{\parskip}{0pt}}{\end{enumerate}}
\newcommand{\TODO}[1]{\todo[backgroundcolor=none, linecolor=red]{#1}}
\newcommand{\TODOimg}[1]{\todo[inline, \bg=gray]{\textbf{IMG}: #1}}
\newcommand{\TODOcontent}[1]{\todo[\bg=red, linecolor=red]{\textbf{CON}: #1}}
\newcommand{\TODOfig}[1]{\todo[inline, \bg=orange]{\textbf{FIG}: #1}}
% =========================== Shortcuts ===========================

\documentclass{article}
\title{\vspace{-3cm} \\ \myTitle}
\author{\myAuthor}
\date{\myDateUpdated}

\usepackage{enumitem}	% Used for enumalpha environment

\usepackage{hyperref}   % Used for adding PDF metadata

\hypersetup{
    colorlinks=true,
	urlcolor=blue,		% URL colors (\href{}{})
	linkcolor=blue,
    pdfborder={0 0 0},
    pdftitle={\myTitle},
    pdfauthor={\myAuthor},
    pdfsubject={\mySubject},
    pdfkeywords={\myKeywords}
}

% Prevent word hyphenation
\tolerance=1
\emergencystretch=\maxdimen
\hyphenpenalty=10000
\hbadness=10000

\begin{document}        % Official beginning of the document.
\maketitle              % Make title page

\section{How This Guide is Formatted}
This guide is made using \LaTeX{}, a typesetting document format, mainly used
for large documents like STEM documents, theses, and research papers. To read
more about what I think about \LaTeX{}, you
\href{https://husseinesmail.xyz/articles/is-latex-better.html}{can go here}.

\section{Format}
\noindent Course
\begin{itemize*}
	\item Department (Example: ``LE'')
	\item Code (Example: ``EECS'')
	\item Num (Example: ``2001'' for EECS 2001)
	\item Credits (Example: ``3.0'')
	\item Title
	\item URL
	\item Description
	\item Sections
	\begin{itemize*}
		\item Section A
		\begin{itemize*}
			\item Term (Examples: ``F'', ``W'', ``SU'', ``S1'', ...)
			\item Year (Examples: ``2019'', ``2020'', ...). Since F/W terms are on the same page
			\item Code (Examples: ``A'', ``B'', ``M'', ``N'', ...)
			\item Profs
			\item CAT
			\item Meetings
			\begin{itemize*}
				\item Type (Example: LECT, TUTR, etc.)
				\item CAT
				\item Day (MTWRF)
				\item Duration
				\item Location
				\item Num (Example: ``02'' for TUTR 02)
				\item TA
				\item Time (24h)
			\end{itemize*}
		\end{itemize*}
	\end{itemize*}
\end{itemize*}

\end{document}
